\documentclass[10pt, letterpaper]{article}

% Packages:
\usepackage[
    ignoreheadfoot, % set margins without considering header and footer
    top=2 cm, % seperation between body and page edge from the top
    bottom=2 cm, % seperation between body and page edge from the bottom
    left=2 cm, % seperation between body and page edge from the left
    right=2 cm, % seperation between body and page edge from the right
    footskip=1.0 cm, % seperation between body and footer
    % showframe % for debugging 
]{geometry} % for adjusting page geometry
\usepackage{titlesec} % for customizing section titles
\usepackage{tabularx} % for making tables with fixed width columns
\usepackage{array} % tabularx requires this
\usepackage[dvipsnames]{xcolor} % for coloring text
\definecolor{primaryColor}{RGB}{0, 0, 0} % define primary color
\usepackage{enumitem} % for customizing lists
\usepackage{fontawesome5} % for using icons
\usepackage{amsmath} % for math
\usepackage[
    pdftitle={Ahmed MAKROUM's CV},
    pdfauthor={Ahmed MAKROUM},
    pdfcreator={LaTeX with RenderCV},
    colorlinks=true,
    urlcolor=primaryColor
]{hyperref} % for links, metadata and bookmarks
\usepackage[pscoord]{eso-pic} % for floating text on the page
\usepackage{calc} % for calculating lengths
\usepackage{bookmark} % for bookmarks
\usepackage{lastpage} % for getting the total number of pages
\usepackage{changepage} % for one column entries (adjustwidth environment)
\usepackage{paracol} % for two and three column entries
\usepackage{ifthen} % for conditional statements
\usepackage{needspace} % for avoiding page brake right after the section title
\usepackage{iftex} % check if engine is pdflatex, xetex or luatex

% Ensure that generate pdf is machine readable/ATS parsable:
\ifPDFTeX
    \input{glyphtounicode}
    \pdfgentounicode=1
    \usepackage[T1]{fontenc}
    \usepackage[utf8]{inputenc}
    \usepackage{lmodern}
\fi

\usepackage{charter}

% Some settings:
\raggedright
\AtBeginEnvironment{adjustwidth}{\partopsep0pt} % remove space before adjustwidth environment
\pagestyle{empty} % no header or footer
\setcounter{secnumdepth}{0} % no section numbering
\setlength{\parindent}{0pt} % no indentation
\setlength{\topskip}{0pt} % no top skip
\setlength{\columnsep}{0.15cm} % set column seperation
\pagenumbering{gobble} % no page numbering

\titleformat{\section}{\needspace{4\baselineskip}\bfseries\large}{}{0pt}{}[\vspace{1pt}\titlerule]

\titlespacing{\section}{
    % left space:
    -1pt
}{
    % top space:
    0.3 cm
}{
    % bottom space:
    0.2 cm
} % section title spacing

\renewcommand\labelitemi{$\vcenter{\hbox{\small$\bullet$}}$} % custom bullet points
\newenvironment{highlights}{
    \begin{itemize}[
        topsep=0.10 cm,
        parsep=0.10 cm,
        partopsep=0pt,
        itemsep=0pt,
        leftmargin=0 cm + 10pt
    ]
}{
    \end{itemize}
} % new environment for highlights


\newenvironment{highlightsforbulletentries}{
    \begin{itemize}[
        topsep=0.10 cm,
        parsep=0.10 cm,
        partopsep=0pt,
        itemsep=0pt,
        leftmargin=10pt
    ]
}{
    \end{itemize}
} % new environment for highlights for bullet entries

\newenvironment{onecolentry}{
    \begin{adjustwidth}{
        0 cm + 0.00001 cm
    }{
        0 cm + 0.00001 cm
    }
}{
    \end{adjustwidth}
} % new environment for one column entries

\newenvironment{twocolentry}[2][]{
    \onecolentry
    \def\secondColumn{#2}
    \setcolumnwidth{\fill, 4.5 cm}
    \begin{paracol}{2}
}{
    \switchcolumn \raggedleft \secondColumn
    \end{paracol}
    \endonecolentry
} % new environment for two column entries

\newenvironment{threecolentry}[3][]{
    \onecolentry
    \def\thirdColumn{#3}
    \setcolumnwidth{, \fill, 4.5 cm}
    \begin{paracol}{3}
    {\raggedright #2} \switchcolumn
}{
    \switchcolumn \raggedleft \thirdColumn
    \end{paracol}
    \endonecolentry
} % new environment for three column entries

\newenvironment{header}{
    \setlength{\topsep}{0pt}\par\kern\topsep\centering\linespread{1.5}
}{
    \par\kern\topsep
} % new environment for the header

\newcommand{\placelastupdatedtext}{% \placetextbox{<horizontal pos>}{<vertical pos>}{<stuff>}
  \AddToShipoutPictureFG*{% Add <stuff> to current page foreground
    \put(
        \LenToUnit{\paperwidth-2 cm-0 cm+0.05cm},
        \LenToUnit{\paperheight-1.0 cm}
    ){\vtop{{\null}\makebox[0pt][c]{
        \small\color{gray}\textit{Last updated in September 2024}\hspace{\widthof{Last updated in September 2024}}
    }}}%
  }%
}%

% save the original href command in a new command:
\let\hrefWithoutArrow\href

% new command for external links:


\begin{document}
    \newcommand{\AND}{\unskip
        \cleaders\copy\ANDbox\hskip\wd\ANDbox
        \ignorespaces
    }
    \newsavebox\ANDbox
    \sbox\ANDbox{$|$}

    \begin{header}
        \fontsize{25 pt}{25 pt}\selectfont Ahmed MAKROUM

        \vspace{5 pt}

        \normalsize
        \mbox{Casablanca, Maroc}%
        \kern 5.0 pt%
        \AND%
        \kern 5.0 pt%
        \mbox{\hrefWithoutArrow{mailto:ahmedmakroum3@gmail.com}{ahmedmakroum3@gmail.com}}%
        \kern 5.0 pt%
        \AND%
        \kern 5.0 pt%
        \mbox{\hrefWithoutArrow{tel:+212 664716219}{06 64 71 62 19}}%
        \kern 5.0 pt%
        \AND%
        \kern 5.0 pt%
        \mbox{\hrefWithoutArrow{https://makroum.website/}{makroum.website}}%
        \kern 5.0 pt%
        \AND%
        \kern 5.0 pt%
        \mbox{\hrefWithoutArrow{https://www.linkedin.com/in/ahmed-makroum/}{/in/ahmed-makroum}}%
        \kern 5.0 pt%
        \AND%
        \kern 5.0 pt%
        \mbox{\hrefWithoutArrow{https://github.com/ahmedmakroum}{github.com/ahmedmakroum}}%
    \end{header}

    \vspace{5 pt - 0.3 cm}


    \section{Profil}



        
        \begin{onecolentry}
            Ingénieur d'état spécialisé en data engineering, avec une pratique concrète de la création de pipelines de données, de systèmes d’ingestion et de transformation, et de l’utilisation de plateformes cloud. Je cherche à rejoindre une équipe où je pourrai concevoir des architectures de données fiables, en améliorer la performance, tout en continuant à affiner mes compétences à travers des projets exigeants.

        \end{onecolentry}





    
    \section{Experience}


    \begin{twocolentry}{
            Septembre 2025 –  Present
        }
            \textbf{Data Engineer}, Metaverse \end{twocolentry}

        \vspace{0.10 cm}
        \begin{onecolentry}
            \begin{highlights}
                \item Développement d’un pipeline ETL distribué avec Apache Beam sur Google Cloud pour traiter des millions de logs utilisateurs
                \item Structuration de données issues de buckets GCP, rendant possible une analyse comportementale fiable
                \item Conception de dashboards automatisés via Looker, adoptés par l’équipe pour améliorer les prises de décisions stratégiques
            \end{highlights}
        \end{onecolentry}


        \vspace{0.2 cm}
        
        \begin{twocolentry}{
            Mars 2025 – Septembre 2025
        }
            \textbf{Data Engineer}, Allianz Maroc \end{twocolentry}

        \vspace{0.10 cm}
        \begin{onecolentry}
            \begin{highlights}
                \item Conception et déploiement de pipelines de données (NiFi, Spark) pour l’extraction, le nettoyage et le chargement depuis les systèmes d’assurance internes, permettant d’automatiser un traitement auparavant manuel.
                \item Intégration des sources fragmentées dans un entrepôt PostgreSQL, consolidant plusieurs flux métiers.
                \item Création de dashboards avec Metabase, facilitant l’accès aux données pour les utilisateurs métiers et rendant possible une visualisation directe des indicateurs clés.
                \item Développement et maintenance d’une application web de gestion des règlements de santé en architecture microservices (Spring Boot, Next.js), avec gestion fine des rôles et authentification sécurisée, utilisée par les équipes métier au quotidien.
            \end{highlights}
        \end{onecolentry}


        \vspace{0.2 cm}
        
        \begin{twocolentry}{
            Juin 2024 – Août 2024
        }
            \textbf{Data Engineer}, Boti School \end{twocolentry}

        \vspace{0.10 cm}
        \begin{onecolentry}
            \begin{highlights}
                \item Développement d’un pipeline ETL distribué avec Apache Beam sur Google Cloud pour traiter des millions de logs utilisateurs
                \item Structuration de données issues de buckets GCP, rendant possible une analyse comportementale fiable
                \item Conception de dashboards automatisés via Looker, adoptés par l’équipe pour améliorer les prises de décisions stratégiques
            \end{highlights}
        \end{onecolentry}



        \vspace{0.2 cm}

        \begin{twocolentry}{
            Juin 2023 – Août 2023
        }
            \textbf{Développeur Web et Mobile}, 6Solutions\end{twocolentry}

        \vspace{0.10 cm}
        \begin{onecolentry}
            \begin{highlights}
                \item Développement d’une plateforme de consulting multi-services (juridique, médical, financier) 
                \item Front-end réalisé avec Angular, back-end avec Spring Boot
                \item Conception et déploiement de la version mobile de l’application en Flutter
            \end{highlights}
        \end{onecolentry}



        \vspace{0.2 cm}

        \begin{twocolentry}{
            Juillet 2023 – Août 2023
        }
            \textbf{Développeur Jeux Vidéo}, Finso\end{twocolentry}

        \vspace{0.10 cm}
        \begin{onecolentry}
            \begin{highlights}
                \item Développement d’un jeu éducatif ludique, destiné aux enfants, pour introduire des notions simples de gestion et de stratégie
                \item Réalisation sous Unity avec un gameplay interactif, des animations soignées et une interface adaptée à un jeune public
            \end{highlights}
        \end{onecolentry}



    \section{Education}



        
        \begin{twocolentry}{
            Sept 2000 – May 2005
        }
            \textbf{University of Pennsylvania}, BS in Computer Science\end{twocolentry}

        \vspace{0.10 cm}
        \begin{onecolentry}
            \begin{highlights}
                \item GPA: 3.9/4.0 (\href{https://example.com}{a link to somewhere})
                \item \textbf{Coursework:} Computer Architecture, Comparison of Learning Algorithms, Computational Theory
            \end{highlights}
        \end{onecolentry}




    
    \section{Projects}



        
        \begin{twocolentry}{
            \href{https://github.com/sinaatalay/rendercv}{github.com/name/repo}
        }
            \textbf{Multi-User Drawing Tool}\end{twocolentry}

        \vspace{0.10 cm}
        \begin{onecolentry}
            \begin{highlights}
                \item Developed an electronic classroom where multiple users can simultaneously view and draw on a "chalkboard" with each person's edits synchronized
                \item Tools Used: C++, MFC
            \end{highlights}
        \end{onecolentry}


        \vspace{0.2 cm}

        \begin{twocolentry}{
            \href{https://github.com/sinaatalay/rendercv}{github.com/name/repo}
        }
            \textbf{Synchronized Desktop Calendar}\end{twocolentry}

        \vspace{0.10 cm}
        \begin{onecolentry}
            \begin{highlights}
                \item Developed a desktop calendar with globally shared and synchronized calendars, allowing users to schedule meetings with other users
                \item Tools Used: C\#, .NET, SQL, XML
            \end{highlights}
        \end{onecolentry}


        \vspace{0.2 cm}

        \begin{twocolentry}{
            2002
        }
            \textbf{Custom Operating System}\end{twocolentry}

        \vspace{0.10 cm}
        \begin{onecolentry}
            \begin{highlights}
                \item Built a UNIX-style OS with a scheduler, file system, text editor, and calculator
                \item Tools Used: C
            \end{highlights}
        \end{onecolentry}



    
    \section{Skills}



        
        \begin{onecolentry}
            \textbf{Languages:} C++, C, Java, Objective-C, C\#, SQL, JavaScript
        \end{onecolentry}

        \vspace{0.2 cm}

        \begin{onecolentry}
            \textbf{Technologies:} .NET, Microsoft SQL Server, XCode, Interface Builder
        \end{onecolentry}


    

\end{document}