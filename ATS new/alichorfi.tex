\documentclass[10pt, letterpaper]{article}

% Packages:
\usepackage[
    ignoreheadfoot, % set margins without considering header and footer
    top=1.3 cm, % seperation between body and page edge from the top
    bottom=1.3 cm, % seperation between body and page edge from the bottom
    left=2 cm, % seperation between body and page edge from the left
    right=2 cm, % seperation between body and page edge from the right
    footskip=1.0 cm, % seperation between body and footer
    % showframe % for debugging 
]{geometry} % for adjusting page geometry
\usepackage{titlesec} % for customizing section titles
\usepackage{tabularx} % for making tables with fixed width columns
\usepackage{array} % tabularx requires this
\usepackage[dvipsnames]{xcolor} % for coloring text
\definecolor{primaryColor}{RGB}{0, 0, 0} % define primary color
\usepackage{enumitem} % for customizing lists
\usepackage{fontawesome5} % for using icons
\usepackage{amsmath} % for math
\usepackage[
    pdftitle={Ahmed MAKROUM's CV},
    pdfauthor={Ahmed MAKROUM},
    pdfcreator={LaTeX with RenderCV},
    colorlinks=true,
    urlcolor=primaryColor
]{hyperref} % for links, metadata and bookmarks
\usepackage[pscoord]{eso-pic} % for floating text on the page
\usepackage{calc} % for calculating lengths
\usepackage{bookmark} % for bookmarks
\usepackage{lastpage} % for getting the total number of pages
\usepackage{changepage} % for one column entries (adjustwidth environment)
\usepackage{paracol} % for two and three column entries
\usepackage{ifthen} % for conditional statements
\usepackage{needspace} % for avoiding page brake right after the section title
\usepackage{iftex} % check if engine is pdflatex, xetex or luatex
\usepackage{graphicx} % for including images

% Ensure that generate pdf is machine readable/ATS parsable:
\ifPDFTeX
    \input{glyphtounicode}
    \pdfgentounicode=1
    \usepackage[T1]{fontenc}
    \usepackage[utf8]{inputenc}
    \usepackage{lmodern}
\fi

\usepackage{charter}

% Some settings:
\raggedright
\AtBeginEnvironment{adjustwidth}{\partopsep0pt} % remove space before adjustwidth environment
\pagestyle{empty} % no header or footer
\setcounter{secnumdepth}{0} % no section numbering
\setlength{\parindent}{0pt} % no indentation
\setlength{\topskip}{0pt} % no top skip
\setlength{\columnsep}{0.15cm} % set column seperation
\pagenumbering{gobble} % no page numbering

\titleformat{\section}{\needspace{4\baselineskip}\bfseries\large}{}{0pt}{}[\vspace{1pt}\titlerule]

\titlespacing{\section}{
    % left space:
    -1pt
}{
    % top space:
    0.3 cm
}{
    % bottom space:
    0.2 cm
} % section title spacing

\renewcommand\labelitemi{$\vcenter{\hbox{\small$\bullet$}}$} % custom bullet points
\newenvironment{highlights}{
    \begin{itemize}[
        topsep=0.10 cm,
        parsep=0.10 cm,
        partopsep=0pt,
        itemsep=0pt,
        leftmargin=0 cm + 10pt
    ]
}{
    \end{itemize}
} % new environment for highlights


\newenvironment{highlightsforbulletentries}{
    \begin{itemize}[
        topsep=0.10 cm,
        parsep=0.10 cm,
        partopsep=0pt,
        itemsep=0pt,
        leftmargin=10pt
    ]
}{
    \end{itemize}
} % new environment for highlights for bullet entries

\newenvironment{onecolentry}{
    \begin{adjustwidth}{
        0 cm + 0.00001 cm
    }{
        0 cm + 0.00001 cm
    }
}{
    \end{adjustwidth}
} % new environment for one column entries

\newenvironment{twocolentry}[2][]{
    \onecolentry
    \def\secondColumn{#2}
    \setcolumnwidth{\fill, 4.5 cm}
    \begin{paracol}{2}
}{
    \switchcolumn \raggedleft \secondColumn
    \end{paracol}
    \endonecolentry
} % new environment for two column entries

\newenvironment{threecolentry}[3][]{
    \onecolentry
    \def\thirdColumn{#3}
    \setcolumnwidth{, \fill, 4.5 cm}
    \begin{paracol}{3}
    {\raggedright #2} \switchcolumn
}{
    \switchcolumn \raggedleft \thirdColumn
    \end{paracol}
    \endonecolentry
} % new environment for three column entries

\newenvironment{header}{
    \setlength{\topsep}{0pt}\par\kern\topsep\centering\linespread{1.5}
}{
    \par\kern\topsep
} % new environment for the header

\newcommand{\placelastupdatedtext}{% \placetextbox{<horizontal pos>}{<vertical pos>}{<stuff>}
  \AddToShipoutPictureFG*{% Add <stuff> to current page foreground
    \put(
        \LenToUnit{\paperwidth-2 cm-0 cm+0.05cm},
        \LenToUnit{\paperheight-1.0 cm}
    ){\vtop{{\null}\makebox[0pt][c]{
        \small\color{gray}\textit{Last updated in September 2024}\hspace{\widthof{Last updated in September 2024}}
    }}}%
  }%
}%

% save the original href command in a new command:
\let\hrefWithoutArrow\href

% new command for external links:


\begin{document}
    \newcommand{\AND}{\unskip
        \cleaders\copy\ANDbox\hskip\wd\ANDbox
        \ignorespaces
    }
    \newsavebox\ANDbox
    \sbox\ANDbox{$|$}

    \begin{header}
        \begin{minipage}[c]{0.7\textwidth}
            \centering
            \fontsize{25 pt}{25 pt}\selectfont Ali CHORFI

            \vspace{5 pt}

            \normalsize
            \mbox{Casablanca, Maroc}%
            \kern 5.0 pt%
            \AND%
            \kern 5.0 pt%
            \mbox{\hrefWithoutArrow{mailto:alichorfi20@gmail.com}{alichorfi20@gmail.com}}%
            \kern 5.0 pt%
            \AND%
            \kern 5.0 pt%
            \mbox{\hrefWithoutArrow{tel:+212 648109759}{06 48 10 97 59}}%
            \kern 5.0 pt%
            \AND%
            \kern 5.0 pt%
            \mbox{\hrefWithoutArrow{https://www.linkedin.com/in/ali-chorfi-35251a26a/}{/in/ali-chorfi}}%
            \kern 5.0 pt%
            \AND%
        \end{minipage}%
        \hfill%
        \begin{minipage}[c]{0.25\textwidth}
            \raggedleft
            \includegraphics[width=3cm,height=3cm,keepaspectratio]{../images/ahmed.jpg}
        \end{minipage}
    \end{header}

    \vspace{5 pt - 0.3 cm}


    \section{Profil}



        
        \begin{onecolentry}
Ingénieur d’État en informatique et réseaux, je développe des applications web et mobiles complètes, côté client et serveur, en privilégiant la clarté, l’utilisabilité et la fiabilité, avec une expérience des outils cloud, de l’hébergement et du CI/CD.
À la recherche d’un poste pour contribuer à des projets concrets.
        \end{onecolentry}





    
    \section{Expérience}




    




        \vspace{0.2 cm}

        \begin{twocolentry}{
            Mars 2025 – Sept 2025
        }
            \textbf{Stage – Développeur Web}, 6Solutions\end{twocolentry}

        \vspace{0.10 cm}
        \begin{onecolentry}
            \begin{highlights}
                \item Développé une plateforme de consulting multi-services (juridique, médical, financier), offrant une solution centralisée pour différents besoins clients 
                \item Réalisé le front-end avec Angular et le back-end avec Spring Boot, assurant une architecture web cohérente
                \item Conçu et déployé la version mobile de l’application en Flutter, permettant un accès fluide aux services depuis n’importe quel appareil
    
            \end{highlights}
        \end{onecolentry}



        \vspace{0.2 cm}

        \begin{twocolentry}{
            Juil 2024 – Août 2024
        }
            \textbf{Stage – Développeur Jeux Vidéo}, Finso\end{twocolentry}

        \vspace{0.10 cm}
        \begin{onecolentry}
            \begin{highlights}
                \item Développé un jeu éducatif ludique pour enfants, introduisant des notions de gestion et de stratégie de manière interactive
                \item Réalisé le projet sous Unity, avec gameplay interactif, animations soignées et interface adaptée, assurant une expérience engageante pour le jeune public
            \end{highlights}
        \end{onecolentry}



    \vspace{0.2 cm}
        
        \begin{twocolentry}{
            Juil 2025 – Août 2023
        }
            \textbf{Stage – Développeur Mobile}, Esith \end{twocolentry}

        \vspace{0.10 cm}
        \begin{onecolentry}
            \begin{highlights}
\item Conçu et développé l’application mobile officielle de l’école ESITH avec React Native, destinée aux étudiants, intégrant une authentification sécurisée et un accès centralisé aux informations institutionnelles.
\item Implémenté un backend serverless basé sur Firebase (Authentication, Firestore), assurant la gestion des comptes étudiants, des contenus applicatifs et la mise à jour dynamique des données.
\item Développé des modules fonctionnels incluant la présentation de l’établissement, une section FAQ, la gestion et la mise en avant des clubs étudiants, ainsi que des contenus informatifs accessibles en libre-service.
\item Conçu une interface mobile performante et intuitive, optimisée pour une utilisation quotidienne par les étudiants et alignée avec l’identité de l’établissement.


            \end{highlights}
        \end{onecolentry}




    \section{Formation}



        
        \begin{twocolentry}{
            Sept 2020 – Mars 2025
        }
            \textbf{EMSI – École Marocaine des Sciences de l’Ingénieur}, Diplôme d’Ingénieur en Informatique et Réseaux option MIAGE \end{twocolentry}


        \end{onecolentry}




    
   




    
    \section{Compétences}


        \begin{onecolentry}
            \textbf{Data Engineering \& Cloud:} Spark, NiFi, dbt, Databricks, Airflow, Kafka • GCP, AWS, Docker, Terraform, DataOps
        \end{onecolentry}

                \vspace{0.1 cm}
        \begin{onecolentry}
            \textbf{Programmation \& Bases de Données:} Python, SQL, Java • PostgreSQL, Hive, MongoDB, Cassandra
        \end{onecolentry}






    

\end{document}