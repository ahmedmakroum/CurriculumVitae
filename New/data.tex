\documentclass[10pt, letterpaper]{article}

% Packages:
\usepackage[
    ignoreheadfoot, % set margins without considering header and footer
    top=1.3 cm, % seperation between body and page edge from the top
    bottom=1.3 cm, % seperation between body and page edge from the bottom
    left=2 cm, % seperation between body and page edge from the left
    right=2 cm, % seperation between body and page edge from the right
    footskip=1.0 cm, % seperation between body and footer
    % showframe % for debugging 
]{geometry} % for adjusting page geometry
\usepackage{titlesec} % for customizing section titles
\usepackage{tabularx} % for making tables with fixed width columns
\usepackage{array} % tabularx requires this
\usepackage[dvipsnames]{xcolor} % for coloring text
\definecolor{primaryColor}{RGB}{0, 0, 0} % define primary color
\usepackage{enumitem} % for customizing lists
\usepackage{fontawesome5} % for using icons
\usepackage{amsmath} % for math
\usepackage[
    pdftitle={Ahmed MAKROUM's CV},
    pdfauthor={Ahmed MAKROUM},
    pdfcreator={LaTeX with RenderCV},
    colorlinks=true,
    urlcolor=primaryColor
]{hyperref} % for links, metadata and bookmarks
\usepackage[pscoord]{eso-pic} % for floating text on the page
\usepackage{calc} % for calculating lengths
\usepackage{bookmark} % for bookmarks
\usepackage{lastpage} % for getting the total number of pages
\usepackage{changepage} % for one column entries (adjustwidth environment)
\usepackage{paracol} % for two and three column entries
\usepackage{ifthen} % for conditional statements
\usepackage{needspace} % for avoiding page brake right after the section title
\usepackage{iftex} % check if engine is pdflatex, xetex or luatex

% Ensure that generate pdf is machine readable/ATS parsable:
\ifPDFTeX
    \input{glyphtounicode}
    \pdfgentounicode=1
    \usepackage[T1]{fontenc}
    \usepackage[utf8]{inputenc}
    \usepackage{lmodern}
\fi

\usepackage{charter}

% Some settings:
\raggedright
\AtBeginEnvironment{adjustwidth}{\partopsep0pt} % remove space before adjustwidth environment
\pagestyle{empty} % no header or footer
\setcounter{secnumdepth}{0} % no section numbering
\setlength{\parindent}{0pt} % no indentation
\setlength{\topskip}{0pt} % no top skip
\setlength{\columnsep}{0.15cm} % set column seperation
\pagenumbering{gobble} % no page numbering

\titleformat{\section}{\needspace{4\baselineskip}\bfseries\large}{}{0pt}{}[\vspace{1pt}\titlerule]

\titlespacing{\section}{
    % left space:
    -1pt
}{
    % top space:
    0.3 cm
}{
    % bottom space:
    0.2 cm
} % section title spacing

\renewcommand\labelitemi{$\vcenter{\hbox{\small$\bullet$}}$} % custom bullet points
\newenvironment{highlights}{
    \begin{itemize}[
        topsep=0.10 cm,
        parsep=0.10 cm,
        partopsep=0pt,
        itemsep=0pt,
        leftmargin=0 cm + 10pt
    ]
}{
    \end{itemize}
} % new environment for highlights


\newenvironment{highlightsforbulletentries}{
    \begin{itemize}[
        topsep=0.10 cm,
        parsep=0.10 cm,
        partopsep=0pt,
        itemsep=0pt,
        leftmargin=10pt
    ]
}{
    \end{itemize}
} % new environment for highlights for bullet entries

\newenvironment{onecolentry}{
    \begin{adjustwidth}{
        0 cm + 0.00001 cm
    }{
        0 cm + 0.00001 cm
    }
}{
    \end{adjustwidth}
} % new environment for one column entries

\newenvironment{twocolentry}[2][]{
    \onecolentry
    \def\secondColumn{#2}
    \setcolumnwidth{\fill, 4.5 cm}
    \begin{paracol}{2}
}{
    \switchcolumn \raggedleft \secondColumn
    \end{paracol}
    \endonecolentry
} % new environment for two column entries

\newenvironment{threecolentry}[3][]{
    \onecolentry
    \def\thirdColumn{#3}
    \setcolumnwidth{, \fill, 4.5 cm}
    \begin{paracol}{3}
    {\raggedright #2} \switchcolumn
}{
    \switchcolumn \raggedleft \thirdColumn
    \end{paracol}
    \endonecolentry
} % new environment for three column entries

\newenvironment{header}{
    \setlength{\topsep}{0pt}\par\kern\topsep\centering\linespread{1.5}
}{
    \par\kern\topsep
} % new environment for the header

\newcommand{\placelastupdatedtext}{% \placetextbox{<horizontal pos>}{<vertical pos>}{<stuff>}
  \AddToShipoutPictureFG*{% Add <stuff> to current page foreground
    \put(
        \LenToUnit{\paperwidth-2 cm-0 cm+0.05cm},
        \LenToUnit{\paperheight-1.0 cm}
    ){\vtop{{\null}\makebox[0pt][c]{
        \small\color{gray}\textit{Last updated in September 2024}\hspace{\widthof{Last updated in September 2024}}
    }}}%
  }%
}%

% save the original href command in a new command:
\let\hrefWithoutArrow\href

% new command for external links:


\begin{document}
    \newcommand{\AND}{\unskip
        \cleaders\copy\ANDbox\hskip\wd\ANDbox
        \ignorespaces
    }
    \newsavebox\ANDbox
    \sbox\ANDbox{$|$}

    \begin{header}
        \fontsize{25 pt}{25 pt}\selectfont Ahmed MAKROUM

        \vspace{5 pt}

        \normalsize
        \mbox{Casablanca, Maroc}%
        \kern 5.0 pt%
        \AND%
        \kern 5.0 pt%
        \mbox{\hrefWithoutArrow{mailto:ahmedmakroum3@gmail.com}{ahmedmakroum3@gmail.com}}%
        \kern 5.0 pt%
        \AND%
        \kern 5.0 pt%
        \mbox{\hrefWithoutArrow{tel:+212 664716219}{06 64 71 62 19}}%
        \kern 5.0 pt%
        \AND%
        \kern 5.0 pt%
        \mbox{\hrefWithoutArrow{https://makroum.website/}{makroum.website}}%
        \kern 5.0 pt%
        \AND%
        \kern 5.0 pt%
        \mbox{\hrefWithoutArrow{https://www.linkedin.com/in/ahmed-makroum/}{/in/ahmed-makroum}}%
        \kern 5.0 pt%
        \AND%
        \kern 5.0 pt%
        \mbox{\hrefWithoutArrow{https://github.com/ahmedmakroum}{github.com/ahmedmakroum}}%
    \end{header}

    \vspace{5 pt - 0.3 cm}


    \section{Profil}



        
        \begin{onecolentry}
            Ingénieur d'état spécialisé en data engineering, avec une pratique concrète de la création de pipelines de données, de systèmes d’ingestion et de transformation, et de l’utilisation de plateformes cloud. Je cherche à rejoindre une équipe où je pourrai concevoir des architectures de données fiables, en améliorer la performance, tout en continuant à affiner mes compétences à travers des projets exigeants.

        \end{onecolentry}





    
    \section{Expérience}


    \begin{twocolentry}{
            Sept 2025 –  Present
        }
            \textbf{Data Engineer}, Metaverse \end{twocolentry}

        \vspace{0.10 cm}
        \begin{onecolentry}
            \begin{highlights}
                \item Créé des plans d’infrastructure data pour des clients avec GCP, AWS et des systèmes internes, facilitant la mise en place et l’uniformité des projets 
                \item Mis en place l’automatisation des déploiements avec Terraform, réduisant le travail manuel et rendant les environnements plus faciles à gérer et à étendre  

            \end{highlights}
        \end{onecolentry}


        \vspace{0.2 cm}
        
        \begin{twocolentry}{
            Mars 2025 – Sept 2025
        }
            \textbf{Stage – Data Engineer}, Allianz Maroc \end{twocolentry}

        \vspace{0.10 cm}
        \begin{onecolentry}
            \begin{highlights}
                \item Conçu et déployé des pipelines de données (NiFi, Spark) pour extraire, nettoyer et charger les données des systèmes d’assurance internes, automatisant un traitement auparavant manuel
                \item Intégré des sources fragmentées dans un entrepôt PostgreSQL, consolidant plusieurs flux métiers
                \item Créé des dashboards avec Metabase, facilitant l’accès aux données et la visualisation directe des indicateurs clé pour les utilisateurs métiers
                \item Développé et maintenu une application web de gestion des règlements de santé en architecture microservices (Spring Boot, Next.js), assurant une gestion fine des rôles et une authentification sécurisée, utilisée quotidiennement par les équipes métier

            \end{highlights}
        \end{onecolentry}


        \vspace{0.2 cm}
        
        \begin{twocolentry}{
            Juin 2024 – Août 2024
        }
            \textbf{Stage – Data Engineer}, Boti School \end{twocolentry}

        \vspace{0.10 cm}
        \begin{onecolentry}
            \begin{highlights}
                \item Conçu et déployé un pipeline ETL distribué avec Apache Beam sur Google Cloud, capable de traiter des millions de logs utilisateurs de manière fiable et performante.  
                \item Structuré et optimisé des données issues de buckets GCP, permettant une analyse comportementale exploitable
                \item Développé des dashboards automatisés via Looker, adoptés par l’équipe pour soutenir les décisions stratégiques
                
            \end{highlights}
        \end{onecolentry}



        \vspace{0.2 cm}

        \begin{twocolentry}{
            Juin 2023 – Août 2023
        }
            \textbf{Stage – Développeur Web et Mobile}, 6Solutions\end{twocolentry}

        \vspace{0.10 cm}
        \begin{onecolentry}
            \begin{highlights}
                \item Développé une plateforme de consulting multi-services (juridique, médical, financier), offrant une solution centralisée pour différents besoins clients 
                \item Réalisé le front-end avec Angular et le back-end avec Spring Boot, assurant une architecture web cohérente
                \item Conçu et déployé la version mobile de l’application en Flutter, permettant un accès fluide aux services depuis n’importe quel appareil
    
            \end{highlights}
        \end{onecolentry}



        \vspace{0.2 cm}

        \begin{twocolentry}{
            Juillet 2023 – Août 2023
        }
            \textbf{Stage –Développeur Jeux Vidéo}, Finso\end{twocolentry}

        \vspace{0.10 cm}
        \begin{onecolentry}
            \begin{highlights}
                \item Développé un jeu éducatif ludique pour enfants, introduisant des notions de gestion et de stratégie de manière interactive
                \item Réalisé le projet sous Unity, avec gameplay interactif, animations soignées et interface adaptée, assurant une expérience engageante pour le jeune public
            \end{highlights}
        \end{onecolentry}



    \section{Formation}



        
        \begin{twocolentry}{
            Sept 2020 – Mars 2025
        }
            \textbf{EMSI – École Marocaine des Sciences de l’Ingénieur}, Diplôme d’Ingénieur en Informatique et Réseaux option MIAGE \end{twocolentry}

        \vspace{0.10 cm}
        \begin{onecolentry}
            \begin{highlights}
                \item Méthodes Informatiques Appliquées à la Gestion des Entreprises 

            \end{highlights}
        \end{onecolentry}




    
   




    
    \section{Compétences}



        
        \begin{onecolentry}
            \textbf{Programmation \& Données:} Python, SQL, Java • PostgreSQL, MongoDB, Cassandra
        \end{onecolentry}

        \vspace{0.2 cm}

        \begin{onecolentry}
            \textbf{Data Engineering \& Cloud:} ETL, Big Data (Hadoop, Spark), Kafka • GCP, AWS, Docker, Terraform, CI/CD, Linux
        \end{onecolentry}


    

\end{document}